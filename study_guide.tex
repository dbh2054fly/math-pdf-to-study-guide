
\documentclass[12pt]{article}
\usepackage{amsmath, amsthm, amssymb}
\usepackage{geometry}
\usepackage{hyperref}
\geometry{letterpaper, margin=1in}
\title{Study Guide}
\author{Generated by the OCR Pipeline}
\date{\today}
\begin{document}
\maketitle
\tableofcontents
\newpage

\section{Course Material Summary}
 systems of linear equations are a powerful tool for algebraic and non - algebraic problems . 
 they can be used to solve algebraic and non - algebraic problems . 
 they can also be used as a tool for algebraic and non - algebraic problems .    
 linear equations are a powerful tool for algebraic and non - algebraic problems . 
 they can be used to solve algebraic and non - algebraic problems . 
 they can also be used as a tool for algebraic and non - algebraic problems .    
 linear equations are a powerful tool for algebraic and

 ordered n-tuples of real numbers (x1,x2,x3,x4,x5 ) are geometric spaces that can be used to /abel various objects of interest , like solutions to systems of equations . 
 we make definitions and state theorems that apply to any ordered n-tuples of real numbers , but we only draw pictures for ordered n-tuples of real numbers ( x1,x2,x3,x4,x5 ) and ordered n-tuples of real numbers ( x1,x2,x3,x4,x5 ) . 

 what does the solution set of a linear equation look like ? a.)x+y +z=1 ~~~ a4 plane in space: (ew) = ( this is the implicit equation of the plane.b.)x+y =1 ~~~ a line in the plane: y = 1x x `| this is called the implicit equation of the line.c.)x+y +z=1 ~~~ a4 plane in space: (ew) = ( this is the implicit equation of the plane.d.)x+y =1 ~~~ a line in the plane: y = 1x

 what does the solution set of a linear equation look like ? xxty+z+w=1™% a 3-plane in 4-space ... [not pictured here] 
 everybody get out your gadgets! 
 everybody get out your gadgets! 
 everybody get out your gadgets! 
 everybody get out your gadgets! 
 everybody get out your gadgets! 
 everybody get out your gadgets! 
 everybody get out your gadgets! 
 everybody get out your gadgets! 
 everybody get out your gadgets! 
 everybody get out your gadgets! 
 everybody get out your gadgets! 
 everybody get

 in this paper , we study the problem of finding all the solutions of equivalent (systems of) equations . 
 we show that finding all the solutions of a system of equations means finding a parametric form: a labeling by some r”. in other words , they are equivalent (systems of) equations .    
 [ multiblock footnote omitted ]    [ multiblock footnote omitted ]    [ multiblock footnote omitted ]    [ multiblock footnote omitted ]    [ multiblock footnote omitted ]    [ multiblock footnote omitted ]  

\end{document}
